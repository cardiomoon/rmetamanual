\usepackage{sectsty}
\partfont{\sffamily\color{magenta}}
\chapterfont{\sffamily\color{magenta}}
\sectionfont{\sffamily\color{magenta}}
\subsectionfont{\sffamily\color{magenta}}
\subsubsectionfont{\sffamily\color{magenta}}
\usepackage[onehalfspacing]{setspace}
\usepackage{pdflscape}
\usepackage{colortbl}
\usepackage[table]{xcolor}
\newcolumntype{U}{>{\columncolor[gray]{0.8}}c}
\usepackage{tabularx,booktabs}
\usepackage{boxedminipage}
\usepackage{graphicx}
\usepackage{rotating}
\usepackage{rotfloat}
\usepackage{booktabs}
\usepackage{longtable}
\usepackage{subfigure}
\usepackage{wrapfig}
\usepackage{lipsum}
\usepackage{multirow}
\usepackage[
singlelinecheck=false
]{caption}
\usepackage{makeidx}
\usepackage{color}
%\usepackage[usenames, dvipsnames]{color}
\usepackage{framed}
\usepackage{showidx}
\usepackage[absolute]{textpos}
\usepackage{fancyhdr}
\DeclareGraphicsExtensions{.pdf,.png,.jpg}
\makeindex
\setlength{\parindent}{0pt}
\definecolor{shadecolor}{rgb}{0.9,1,1}
\textblockorigin{0mm}{0mm}
\pagestyle{fancy}
% with this we ensure that the chapter and section
% headings are in lowercase.

\renewcommand{\chaptermark}[1]{\markboth{\textbf{\thechapter\ \chaptername} \color{gray}$\bullet $ \color{black} \ #1}{}}
%\renewcommand{\sectionmark}[1]{\markright{\thesection\ #1}}
\renewcommand{\sectionmark}[1]{\markright{#1}{}}

\fancyhf{}  % delete current setting for header and footer
% \fancyfoot[LE,RO]{\sffamily\color{magenta}\thepage}
 \fancyfoot[LE]{\sffamily\textbf\small\thepage}
 \fancyfoot[RO]{\sffamily\textbf\small\leftmark\hspace{0.5cm}\thepage}
\renewcommand{\headrulewidth}{0pt}
\renewcommand{\footrulewidth}{0.5pt}
\addtolength{\headheight}{0.7pt} % make space for the rule
\fancypagestyle{plain}{%
   \fancyhead{} % get rid of headers on plain pages
   \renewcommand{\headrulewidth}{0pt} % and the line
}



